\section{scsum1.c File Reference}
\label{scsum1_8c}\index{scsum1.c@{scsum1.c}}
Takes sum of the absolute values of a complex vector and returns a single precision result. 

{\tt \#include \char`\"{}slu\_\-scomplex.h\char`\"{}}\par
\subsection*{Defines}
\begin{CompactItemize}
\item 
\#define {\bf CX}(I)~cx[(I)-1]\label{scsum1_8c_5a76da95c549c41790389a76e12fdcb5}

\end{CompactItemize}
\subsection*{Functions}
\begin{CompactItemize}
\item 
double {\bf scsum1\_\-} (int $\ast$n, complex $\ast$cx, int $\ast$incx)
\end{CompactItemize}


\subsection{Detailed Description}
Takes sum of the absolute values of a complex vector and returns a single precision result. 

\small\begin{alltt}
     -- LAPACK auxiliary routine (version 2.0) --   
     Univ. of Tennessee, Univ. of California Berkeley, NAG Ltd.,   
     Courant Institute, Argonne National Lab, and Rice University   
     October 31, 1992   
 \end{alltt}\normalsize 
 

\subsection{Function Documentation}
\index{scsum1.c@{scsum1.c}!scsum1_@{scsum1\_\-}}
\index{scsum1_@{scsum1\_\-}!scsum1.c@{scsum1.c}}
\subsubsection{\setlength{\rightskip}{0pt plus 5cm}double scsum1\_\- (int $\ast$ {\em n}, complex $\ast$ {\em cx}, int $\ast$ {\em incx})}\label{scsum1_8c_5d1c88cf76514de82b3ab994dd6990f4}


\small\begin{alltt}
    Purpose   
    =======\end{alltt}\normalsize 


\small\begin{alltt}    SCSUM1 takes the sum of the absolute values of a complex   
    vector and returns a single precision result.\end{alltt}\normalsize 


\small\begin{alltt}    Based on SCASUM from the Level 1 BLAS.   
    The change is to use the 'genuine' absolute value.\end{alltt}\normalsize 


\small\begin{alltt}    Contributed by Nick Higham for use with CLACON.\end{alltt}\normalsize 


\small\begin{alltt}    Arguments   
    =========\end{alltt}\normalsize 


\small\begin{alltt}    N       (input) INT
            The number of elements in the vector CX.\end{alltt}\normalsize 


\small\begin{alltt}    CX      (input) COMPLEX array, dimension (N)   
            The vector whose elements will be summed.\end{alltt}\normalsize 


\small\begin{alltt}    INCX    (input) INT
            The spacing between successive values of CX.  INCX > 0.\end{alltt}\normalsize 


\small\begin{alltt}    ===================================================================== 
\end{alltt}\normalsize 
 